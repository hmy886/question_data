
试一试:生活中常用“一刹那”形容事情发生得极快,那么一刹那到底是多少秒呢?古代有一本书中这样写道:\\
一刹那者为一念,二十念为一瞬,二十瞬为一弹指,二十弹指为一罗预,二十罗预为一须臾,一日一夜有三十须臾.\\
根据这个叙述,完成以下填空.(结果用分数表示)\\
1 (须臾)= \key{\hspace{4em}}(分钟);\\
1 (罗预)= \key{\hspace{4em}}(分钟);\\
1 (弹指)= \key{\hspace{4em}}(分钟);\\
1 (瞬)= \key{\hspace{4em}}(分钟);\\
1 (刹那)= \key{\hspace{4em}}(分钟)= \key{\hspace{4em}}(秒钟).\\

